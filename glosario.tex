\newglossaryentry{ruteador}{
name={ruteador}, 
description={En una red de computadoras existe un nodo  llamado ruteador, que se encarga de trazar una ruta entre un nodo origen a un nodo destino}
}
\newglossaryentry{matrix}% the label 
{name={matrix},% the term 
 description={a rectangular table of elements},% brief description 
 plural={matrices}% the plural 
} 

\newglossaryentry{MCMC}
{name  = {Markov chain Monte Carlo},
description = {M�todos  de cadenas Markov Montecarlo son una clase de algoritmos para muestras de una distribuci�n de probabilidad sobre la base de la construcci�n  de una cadena de Markov que tiene la distribuci�n deseada como su distribuci�n en equilibrio.}
}
\newglossaryentry{Distribui\'on probabilidad}
{name = {Distribucion de probabilidad},
description = {Una distribuci\'on de probabilidad es una tabla o una ecuaci\'on que liga cada salida de un experimento estadistico con su probabilidad de ocurrecia }
}
\newglossaryentry{teorema de Kuhn-Tucker}
{name={Teorema de Kuhn-Tucker}, 
description={condiciones necesarias y suficientes para que la solución de una programación  no lineal sea óptima. Es una generalización del método de los Multiplicadores de Lagrange}
}

\newglossaryentry{broadcast}
{
name={Broadcast},
description={El direccionamiento de un paquete
a todos los destinos utilizando un código especial en el campo de dirección. Cuando se transmite
un paquete con este código, todas las máquinas de la red lo reciben y procesan. Este modo de operación
se conoce como difusión (broadcasting)}
}

\newglossaryentry{unicast}
{
name={Unicast},
description={La transmisión de punto a punto con un emisor y un receptor se conoce como unidifusión (unicasting)}
}

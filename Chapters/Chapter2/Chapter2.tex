% Chapter 2 : Modelo de Actores

\chapter{Modelo de Actores} % Write in your own chapter title
\label{Chapter2}
\lhead{Cap\'itulo 2. \emph{Modelo de Actores}} % Write in your own chapter title to set the page header


\section{Introduccion al modelo de Actores}

la idea de este capitulo es hablar de un resumen cuando se desarrollo el modelo de actores, qu ees un actor, porque es importante,  con que lo podemos comparar (igual y esta con que es) y por ultimo porque elegir al modelo de actores por (spoiler) hilos.

\section{Definicion de un actor}
Un actor es una la pieza meas primitica en el modelo de actores,e ste actor tiene ciertas responsabilidades: crear un actor, mandar un mensaje, etc....
\section{uso practico del modelo de actores}
Un ejemplo en la vida de un actor que puede reflejarse superior a un primitivo Hilo o threat....


\section{modelo de Actores}


El modelo de actores fue inventado por Carl Hewitt  en 1973: un formalismo
universal de actor modular para la inteligencia artificial, sin embargo la
teoria del modelo de actores se formalizo en 1986 por Carl Hewitt, Henri Baker y
GulAgha en la  tesis \emph{Actors: A Model of Concurrent Computation in
  Distributed Systems}. Los actores son objetos autonomos y concurrentes los
cuales se ejecutan de manera asincrona. El modelo de actores  nos brinda un
mecanismo flexible para  hacer sistemas de software  distribuidos y paralelos

Un actor es una entidad computacional que adopta la filosof\'ia de \emph{todo es un actor}, similar a la filosof\'ia de \emph{todo es un objeto} usada en los lenguajes de programaci\'on orientado a objetos.


Un actor  incorpora 3 capacidades: \textbf{procesar},\textbf{almacenar} y   la capacidad de \textbf{comunicarse} con otros actores, por lo tanto un sistema de actores cuenta con las siguientes caracteristica:

--Comunicaci\'on por mensajes asincronos directos: Un actor puede comunicarse con otro actor siempre y cuando conozca la direccion del actor con quien quiere comunicarse, el paso de mensajes se es asincrono.


--M\'aquinas de estado: Un actor soporta m\'aquinas de estado finita.En cada estado  se puede cambiar el comportamiento del actor para responder ante los mensajes que reciba en el futuro.

--Objetos con bloqueo: Los actores no comparten su estado mutable con otro actores u otro componente que sea importante.

--Libre de bloqueos de concurrencia: como los actores no comparten sus estados mutable, y porque ellos reciben un solo mensaje a la vez, actores nunca necesitan bloquear sus estados antes de reaccionar a un mensaje.

--Paralelismo: en el modelo de actores el paralelismo divide procesos complejos a tareas m\'as peque\~nas para ejecutar las tareas de manera concurrente. [1]



De tal manera que cuando recibe un mensaje un actor  puede hacer lo siguiente:

[Lista  comand]
- Enviar un n\'umero finito de mensajes a otro actor
-Crar numeros finitos de otros actores
-Designar  que es lo que  va a hacer con los mensajes siguiente.



[1] Vaughn Vernon. “Reactive Messaging Patterns With the Actor Model: Applications and Integration in Scala and Akka.”

http://c2.com/cgi/wiki?ActorsModel
http://www.erights.org/history/actors.html
http://doc.akka.io/docs/akka/2.4.1/general/actors.html

\section {Caminatas aleatorias}

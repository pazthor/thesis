\message{ !name(Chapter4.tex)}
\message{ !name(Chapter4.tex) !offset(-2) }
% Chapter 4: Implementacion y descripcion del proyecto realizado

\chapter{Implementacion y descripcion del proyecto realizado} % Write in your own chapter title
\label{Chapter4}
\lhead{Cap\'itulo4. \emph{Practica-muestra/implementacion muestreo de
    redes sociales usando camiantas
    aleatoreas}} % Write in your own chapter title to set the page header


\section{Introduccion del proyecto}

Puesto en marcha la descripccion consisa y detallada de cada tema que
ocupar\'e para desarrollar mi tesis, aca mostrare el como lo hice o lo
que estuve sacando conforme aplicando todo lo que mostre anteriormente
en los capitulos anteriores, no veo exactamente que, pero espero que
en 10(noche del 21 de septiembre) dias lo sepa

\subsection{Diferencias tecnologicas}
\label{sec:tech}

En los puntos que se analizen que son los agentes inteligentes contra
el modelo de actores y por supuesto conocerlas para diferenciarlas y
revisar en que punt podemos se puedan juntar (o no)
\todo{Agentes inteligentes y modelo de actores}

\message{ !name(Chapter4.tex) !offset(-28) }
